OOC research has advanced considerably in the last few decades, but it is still difficult to compare and contrast research findings from different pieces of work.   Part of this is due to the very nature of the research; it comes from all sort of fields from information systems (e.g. \cite{ma07through}) to computer science (e.g. \cite{shneiderman00creating}) to information science (e.g. \cite{preece00online}) to marketing (e.g. \cite{kozinets02field}), and is studied in a wide variety of platforms from Twitter (e.g. \cite{vieweg10microblogging}) to Wikis (e.g. \cite{beschastnikh08wikipedian}) to question-and-answer forums (e.g. \cite{zhang07expertise}).  As a result, different disciplines and different study venues use different language, and different descriptions, making it hard to integrate knowledge gleaned from different origins.

Moreover, the lack of easy translatability between fields and platforms has made the reproducibility of findings very difficult.  A researcher in one field may take certain definitions for granted that are not well understood in another field, or at least are not understood in the same way.  In other words, it's difficult for a researcher who works with Flickr data to understand how certain concepts are operationalized by researchers who work with blogging data. The field has progressed fine up until this point because there is so much research to do in this space, but it is now time to start to build a cohesive theory of online communities and to create knowledge built on top of other knowledge, and to provide standards that allow different researchers to reproduce each others'
findings.

In order to take the field to the next step, it is necessary to develop a standard of communication that will allow different researchers to communicate how and why they performed their analysis and research the way that they did.  One development in other interdisciplinary fields that has helped communication across boundaries is the creation of a uniform methods protocol (e.g. \cite{grimm10odd}).  By creating such standards that describe how data is collected and analyzed, as well as how certain measurements in the theory are operationalized within the data, it is possible to make it easier for different researchers to understand each other's work and to reproduce findings.
