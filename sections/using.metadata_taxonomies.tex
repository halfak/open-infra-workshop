In order to \emph{break into} a new dataset, a researcher will need to discover it and determine know how to make use of it.  Currently, OOC datasets are scattered across various websites on the internet.  They are inconsistently (if at all) documented and the terms used to describe the characteristics of the data differ based on the discipline of the authors.  By gathering and standardizing information about OOC datasets, we can dramatically improve the discover-ability and utility of them.

Classifying OOC datasets so interdisciplinary researchers can discover, access, and use them in collaboration with other scholars requires consistent and agreed upon descriptions. Engaging these challenges in a CSCW workshop will allow us to articulate shared research goals, develop common terminology for describing datasets in generalizable terms, and determine how to document metadata at different descriptive levels so that OOC researchers can use these datasets effectively. 

OOC datasets can be described on three levels: the meta, mezzo and micro.  The meta-level is descriptive information that aids researchers in finding datasets and conducting a preliminary evaluation of their value prior to use.  This level also supports data management\footnote{\url{http://www.dcc.ac.uk/resources/curation-lifecycle-model}}.  The mezzo-level describes the meaning captured in a dataset's content.  Scholars use mezzo-level information to understand OOCs across disciplines, create theories, conduct scholarship, etc.  Last is micro-level dataset information, which granularly describes the contents and structure of a dataset. Tiered metadata schemas allow us to account for different research methods, modes of analysis, storage systems, and disciplinary norms and to support other considerations such as dataset accessibility (e.g. copyright) and research ethics.
