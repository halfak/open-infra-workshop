Despite being easily accessible, open online community (OOC) data can be difficult to use effectively.  In order to access and analyze large amounts of data, researchers must first become familiar with the meaning of data values.  Then they must find a way to obtain and process the datasets to extract their desired vectors of behavior and content.  This process is fraught with problems that are solved (through great difficulty) over and over again by each research team/lab that \emph{breaks into} datasets for a new OOC.  Rarely does the description of methods presented in research papers provide sufficient depth of discussion to enable straightforward replication or extension studies.  Further, those without the technical skills to process large amounts of data effectively will often be prevented left without a starting point.  The result of these factors is a set of missed opportunities around the promise of open data to expedite scientific progress.

In this workshop, we will experiment with strategies -- both technological systems and documentation strategies -- designed to enable our community to more thoroughly reap the benefits of open online data science practice.  We will invite participants to attempt the difficult work of breaking into a new dataset using tools and documentation designed to alleviate common difficulties.  In the months leading up to the workshop, we will prepare and describe several datasets within an open querying service and invite participants to explore these systems and their functionality through the replication and extension of a selected data-intensive research paper from past CSCWs.  During the workshop participants will have the opportunity to explore new tools and datasets and to jump start new studies based on our curated documentation and infrastructure. As we observe and interact with our participants, we hope to learn from their successes and struggles and to use these observations to iteratively improve our tools and documentation protocols.

This work builds on a call to action from a previous CSCW Workshops\cite{goggins14ocdata, morgan15advancing} and ongoing initiatives\footnote{\url{http://www.datafactories.org/}}: to build up shared research infrastructure\cite{wiggins14quality, morgan15advancing_report} to support data and method sharing practices. The workshop organizers come from many different backgrounds and have extensive experience with using OOC data, developing infrastructure to support access to and analysis of OOC data, and building communities of practice around OOC research.

We will use this workshop to achieve three goals:
\begin{enumerate}
\item identify common challenges and novel strategies for making open community research easier to replicate and extend -- specifically targeting protocols for documenting research methods (e.g. the ODD protocol\cite{grimm10odd})
\item inform the design of data management/analysis infrastructures like Quarry, our experimental open querying service\footnote{\url{https://meta.wikimedia.org/wiki/Research:Quarry}}
\item inform the design of metadata indexes like the Open Collaboration Data Factory's wiki\footnote{\url{http://wiki.urbanhogfarm.com/index.php/Main_Page}}
\end{enumerate}

We also hope to foster a community of practice within CSCW around open data management and plan for next steps towards accelerating scientific progress around the study of computer-supported cooperation -- just as past workshops have informed our plans for this workshop proposal.
